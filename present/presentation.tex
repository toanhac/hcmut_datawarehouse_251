\documentclass[aspectratio=169]{beamer}
\usetheme{Madrid}
\usecolortheme{default}

% Packages
\usepackage[utf8]{inputenc}
\usepackage[vietnamese]{babel}
\usepackage{graphicx}
\usepackage{booktabs}
\usepackage{listings}
\usepackage{xcolor}
\usepackage{amssymb}
\usepackage{tikz}
\usetikzlibrary{positioning}
\usepackage{hyperref}
\usepackage{multicol}

% Color definitions
\definecolor{hcmutblue}{RGB}{0,82,155}
\definecolor{codebg}{RGB}{246,248,250}
\definecolor{codegreen}{RGB}{0,96,0}
\definecolor{codepurple}{RGB}{88,0,130}

% Theme customization
\setbeamercolor{structure}{fg=hcmutblue}
\setbeamercolor{title}{fg=white,bg=hcmutblue}
\setbeamercolor{frametitle}{fg=white,bg=hcmutblue}

% Code listing style
\lstset{
    backgroundcolor=\color{codebg},
    basicstyle=\ttfamily\tiny,
    keywordstyle=\color{blue}\bfseries,
    commentstyle=\color{codegreen},
    stringstyle=\color{codepurple},
    breaklines=true,
    showstringspaces=false,
    frame=single,
    numbers=left,
    numberstyle=\tiny\color{gray}
}

% Title information
\title[Bank Churn DWH \& DSS]{Kho dữ liệu \& Hệ hỗ trợ quyết định\\cho bài toán Bank Customer Churn}
\subtitle{Data Warehouse \& Decision Support System}
\author[Hoa Toàn Hạc, Mai Huy Hiệp]{
    Hoa Toàn Hạc (2201917)\\
    Mai Huy Hiệp (2211045)
}
\institute[HCMUT]{
    Trường Đại học Bách khoa TP.HCM\\
    Khoa Khoa học và Kỹ thuật máy tính
}
\date{Tháng 11, 2025}

% Logo
\logo{\includegraphics[height=0.8cm]{../reports/Images/hcmut.png}}

\begin{document}

% ============================================================================
% TITLE SLIDE
% ============================================================================
\begin{frame}
    \titlepage
\end{frame}

% ============================================================================
% TABLE OF CONTENTS
% ============================================================================
\begin{frame}{Nội dung trình bày}
    \tableofcontents
\end{frame}

% ============================================================================
% SECTION 1: INTRODUCTION
% ============================================================================
\section{Giới thiệu}

\begin{frame}{Đặt vấn đề}
    \begin{block}{Tại sao Customer Churn quan trọng?}
        \begin{itemize}
            \item Chi phí thu hút khách hàng mới cao gấp \textbf{5-25 lần} so với giữ chân khách hàng hiện tại
            \item Tỷ lệ churn cao ảnh hưởng trực tiếp đến doanh thu và uy tín ngân hàng
            \item Cần phân tích dữ liệu để \textbf{dự đoán và ngăn chặn} churn
        \end{itemize}
    \end{block}
    
    \vspace{0.5cm}
    
    \begin{block}{Thách thức}
        \begin{itemize}
            \item Dữ liệu khách hàng phân tán, chưa được tổ chức tối ưu
            \item Thiếu công cụ phân tích đa chiều (OLAP)
            \item Chưa có mô hình dự đoán chính xác
            \item Thiếu hệ thống hỗ trợ ra quyết định
        \end{itemize}
    \end{block}
\end{frame}

\begin{frame}{Mục tiêu đề tài}
    \begin{enumerate}
        \item \textbf{Thiết kế Data Warehouse}
        \begin{itemize}
            \item Xây dựng star schema với dimension và fact tables
            \item Tối ưu hóa cho phân tích OLAP
        \end{itemize}
        
        \item \textbf{Xây dựng quy trình ETL}
        \begin{itemize}
            \item Pipeline tự động: Extract → Transform → Load
            \item Feature engineering cho phân tích và modeling
        \end{itemize}
        
        \item \textbf{Phân tích OLAP \& Visualization}
        \begin{itemize}
            \item Truy vấn đa chiều: địa lý, tuổi, thu nhập
            \item Biểu đồ trực quan với Python
        \end{itemize}
        
        \item \textbf{Xây dựng mô hình ML}
        \begin{itemize}
            \item Dự đoán khả năng churn của khách hàng
            \item Đánh giá và tối ưu hóa mô hình
        \end{itemize}
        
        \item \textbf{Hệ hỗ trợ quyết định (DSS)}
        \begin{itemize}
            \item Tích hợp DWH, OLAP, ML thành hệ thống hoàn chỉnh
            \item Dashboard tương tác cho người dùng
        \end{itemize}
    \end{enumerate}
\end{frame}

\begin{frame}{Tổng quan dữ liệu}
    \begin{columns}
        \column{0.5\textwidth}
        \begin{block}{Nguồn dữ liệu}
            \begin{itemize}
                \item \textbf{Dataset}: Bank Customer Churn Modeling
                \item \textbf{Nguồn}: Kaggle
                \item \textbf{Số lượng}: 10,000 khách hàng
                \item \textbf{Số biến}: 14 biến
                \item \textbf{Tỷ lệ churn}: ~20\%
            \end{itemize}
        \end{block}
        
        \column{0.5\textwidth}
        \begin{block}{Các biến chính}
            \textbf{Nhân khẩu học:}
            \begin{itemize}
                \item Age, Gender, Geography
            \end{itemize}
            
            \textbf{Tài chính:}
            \begin{itemize}
                \item CreditScore, Balance
                \item EstimatedSalary
            \end{itemize}
            
            \textbf{Hành vi:}
            \begin{itemize}
                \item Tenure, NumOfProducts
                \item HasCrCard, IsActiveMember
            \end{itemize}
            
            \textbf{Target:}
            \begin{itemize}
                \item Exited (0=Retained, 1=Churned)
            \end{itemize}
        \end{block}
    \end{columns}
\end{frame}

% ============================================================================
% SECTION 2: DATA WAREHOUSE DESIGN
% ============================================================================
\section{Thiết kế kho dữ liệu}

\begin{frame}{Star Schema - Tổng quan}
    \begin{center}
        \begin{tikzpicture}[scale=0.8, every node/.style={transform shape}]
            % Fact table (center)
            \node[rectangle, draw, fill=hcmutblue!20, minimum width=3cm, minimum height=2cm] (fact) at (0,0) {
                \textbf{FACT}\\
                \textbf{customer\_status}\\
                \tiny
                \begin{tabular}{l}
                    fact\_key (PK)\\
                    customer\_key (FK)\\
                    time\_key (FK)\\
                    geo\_key (FK)\\
                    segment\_key (FK)\\
                    balance\\
                    estimated\_salary\\
                    num\_of\_products\\
                    credit\_score\\
                    churn\_flag
                \end{tabular}
            };
            
            % Dimension tables
            \node[rectangle, draw, fill=green!20, minimum width=2cm, minimum height=1.5cm] (dim_customer) at (-5,3) {
                \textbf{DIM}\\
                \textbf{customer}\\
                \tiny
                \begin{tabular}{l}
                    customer\_key (PK)\\
                    customer\_id\\
                    age\\
                    gender\\
                    tenure
                \end{tabular}
            };
            
            \node[rectangle, draw, fill=green!20, minimum width=2cm, minimum height=1.5cm] (dim_geo) at (5,3) {
                \textbf{DIM}\\
                \textbf{geo}\\
                \tiny
                \begin{tabular}{l}
                    geo\_key (PK)\\
                    country
                \end{tabular}
            };
            
            \node[rectangle, draw, fill=green!20, minimum width=2cm, minimum height=1.5cm] (dim_time) at (-5,-3) {
                \textbf{DIM}\\
                \textbf{time}\\
                \tiny
                \begin{tabular}{l}
                    time\_key (PK)\\
                    snapshot\_date\\
                    year\\
                    month\\
                    quarter
                \end{tabular}
            };
            
            \node[rectangle, draw, fill=green!20, minimum width=2cm, minimum height=1.5cm] (dim_segment) at (5,-3) {
                \textbf{DIM}\\
                \textbf{segment}\\
                \tiny
                \begin{tabular}{l}
                    segment\_key (PK)\\
                    age\_group\\
                    income\_group
                \end{tabular}
            };
            
            % Arrows
            \draw[->, thick] (dim_customer) -- (fact);
            \draw[->, thick] (dim_geo) -- (fact);
            \draw[->, thick] (dim_time) -- (fact);
            \draw[->, thick] (dim_segment) -- (fact);
        \end{tikzpicture}
    \end{center}
\end{frame}

\begin{frame}{Bảng chiều - Dimension Tables}
    \begin{columns}[t]
        \column{0.5\textwidth}
        \begin{block}{dim\_customer}
            \footnotesize
            \vspace{-0.1cm}
            Thông tin nhân khẩu học
            \begin{itemize}\setlength\itemsep{-0.2em}
                \item customer\_key (PK)
                \item age, gender, tenure
            \end{itemize}
        \end{block}
        
        \vspace{0.2cm}
        
        \begin{block}{dim\_geo}
            \footnotesize
            \vspace{-0.1cm}
            Thông tin địa lý
            \begin{itemize}\setlength\itemsep{-0.2em}
                \item geo\_key (PK)
                \item country
            \end{itemize}
        \end{block}
        
        \column{0.5\textwidth}
        \begin{block}{dim\_time}
            \footnotesize
            \vspace{-0.1cm}
            Thông tin thời gian
            \begin{itemize}\setlength\itemsep{-0.2em}
                \item time\_key (PK)
                \item snapshot\_date, year, month, quarter
            \end{itemize}
        \end{block}
        
        \vspace{0.2cm}
        
        \begin{block}{dim\_segment}
            \footnotesize
            \vspace{-0.1cm}
            Phân khúc khách hàng
            \begin{itemize}\setlength\itemsep{-0.2em}
                \item segment\_key (PK)
                \item age\_group
                \item income\_group
            \end{itemize}
        \end{block}
    \end{columns}
\end{frame}

\begin{frame}{Grain của Fact Table}
    \begin{alertblock}{Định nghĩa Grain}
        \Large
        \vspace{0.5cm}
        \begin{center}
            \textbf{Một bản ghi cho mỗi khách hàng\\tại một thời điểm snapshot}
        \end{center}
        \vspace{0.5cm}
    \end{alertblock}
    
    \vspace{0.5cm}
    
    \begin{block}{Ý nghĩa}
        \begin{itemize}
            \item Mỗi dòng trong fact table đại diện cho trạng thái của một khách hàng
            \item Tại một thời điểm cụ thể (snapshot date)
            \item Cho phép phân tích theo thời gian nếu có nhiều snapshots
            \item Hỗ trợ tracking sự thay đổi của customer behavior
        \end{itemize}
    \end{block}
\end{frame}

\begin{frame}{Lợi ích của Star Schema}
    \begin{enumerate}
        \item \textbf{Hiệu suất cao}
        \begin{itemize}
            \item Truy vấn OLAP nhanh (simple joins)
            \item Tối ưu cho Business Intelligence tools
        \end{itemize}
        
        \item \textbf{Dễ hiểu và sử dụng}
        \begin{itemize}
            \item Cấu trúc đơn giản, trực quan
            \item Business users dễ dàng viết queries
        \end{itemize}
        
        \item \textbf{Phân tích đa chiều}
        \begin{itemize}
            \item Slice \& Dice theo nhiều chiều
            \item Drill-down, Roll-up linh hoạt
        \end{itemize}
        
        \item \textbf{Khả năng mở rộng}
        \begin{itemize}
            \item Dễ dàng thêm dimension mới
            \item Không ảnh hưởng đến hệ thống OLTP
        \end{itemize}
    \end{enumerate}
\end{frame}

% ============================================================================
% SECTION 3: ETL PROCESS
% ============================================================================
\section{Quy trình ETL}

\begin{frame}{ETL Pipeline - Tổng quan}
    \begin{center}
        \begin{tikzpicture}[node distance=2cm, auto]
            % Nodes
            \node[rectangle, draw, fill=blue!20, minimum width=2.5cm, minimum height=1cm] (extract) {
                \textbf{EXTRACT}\\
                \tiny Read CSV
            };
            
            \node[rectangle, draw, fill=orange!20, minimum width=2.5cm, minimum height=1cm, right=of extract] (transform) {
                \textbf{TRANSFORM}\\
                \tiny Clean \& Engineer
            };
            
            \node[rectangle, draw, fill=green!20, minimum width=2.5cm, minimum height=1cm, right=of transform] (load) {
                \textbf{LOAD}\\
                \tiny Save to DWH
            };
            
            % Arrows
            \draw[->, thick] (extract) -- (transform);
            \draw[->, thick] (transform) -- (load);
        \end{tikzpicture}
    \end{center}
    
    \vspace{0.5cm}
    
    \begin{block}{Các bước chính}
        \begin{enumerate}
            \item \textbf{Extract}: Đọc dữ liệu từ Churn\_Modelling.csv (10,000 records)
            \item \textbf{Transform}:
            \begin{itemize}
                \item Data cleaning: Loại bỏ cột không cần thiết
                \item Feature engineering: Tạo age\_group, income\_group
                \item Build dimensions: 4 dimension tables
                \item Build fact: Join dimensions để lấy surrogate keys
            \end{itemize}
            \item \textbf{Load}: Xuất 5 tables ra CSV hoặc nạp vào PostgreSQL
        \end{enumerate}
    \end{block}
\end{frame}

\begin{frame}[fragile]{Feature Engineering}
    \begin{columns}
        \column{0.5\textwidth}
        \begin{block}{Age Group}
            \begin{lstlisting}[language=Python]
def categorize_age(age):
    if age < 36:
        return 'Young'
    elif age < 56:
        return 'Middle-aged'
    else:
        return 'Senior'

df['age_group'] = df['Age'].apply(
    categorize_age
)
            \end{lstlisting}
        \end{block}
        
        \column{0.5\textwidth}
        \begin{block}{Income Group}
            \begin{lstlisting}[language=Python]
def categorize_income(salary):
    if salary < 50000:
        return 'Low'
    elif salary < 100000:
        return 'Mid'
    else:
        return 'High'

df['income_group'] = df[
    'EstimatedSalary'
].apply(categorize_income)
            \end{lstlisting}
        \end{block}
    \end{columns}
    
    \vspace{0.3cm}
    
    \begin{alertblock}{Kết quả}
        \begin{itemize}
            \item \textbf{dim\_segment}: 9 segments (3 age groups × 3 income groups)
            \item Hỗ trợ phân tích theo phân khúc khách hàng
        \end{itemize}
    \end{alertblock}
\end{frame}

\begin{frame}{Kết quả ETL}
    \begin{table}
        \centering
        \small
        \begin{tabular}{lrr}
            \toprule
            \textbf{Bảng} & \textbf{Số bản ghi} & \textbf{Mô tả} \\
            \midrule
            dim\_customer & 10,000 & Thông tin khách hàng \\
            dim\_geo & 3 & France, Germany, Spain \\
            dim\_time & 1 & Snapshot date (2025-01-01) \\
            dim\_segment & 9 & 3 age groups × 3 income groups \\
            fact\_customer\_status & 10,000 & Measures + Foreign keys \\
            \bottomrule
        \end{tabular}
    \end{table}
    
    \vspace{0.3cm}
    
    \begin{block}{Ưu điểm của quy trình}
        \small
        \begin{itemize}\setlength\itemsep{-0.1em}
            \item \textbf{Tự động hóa}: Toàn bộ bằng Python, chạy lại dễ dàng
            \item \textbf{Modular}: Mỗi bước là function riêng
            \item \textbf{Reproducible}: Kết quả nhất quán
            \item \textbf{Scalable}: Dễ mở rộng cho nhiều snapshots
        \end{itemize}
    \end{block}
\end{frame}

% ============================================================================
% SECTION 4: OLAP ANALYSIS
% ============================================================================
\section{Phân tích OLAP \& Visualization}

\begin{frame}[fragile]{OLAP Queries - Ví dụ}
    \begin{block}{Query 1: Tỷ lệ churn theo quốc gia}
        \begin{lstlisting}[language=SQL]
SELECT 
    g.country,
    COUNT(*) AS total_customers,
    SUM(f.churn_flag) AS churned_customers,
    ROUND(AVG(f.churn_flag) * 100, 2) AS churn_rate_pct
FROM fact_customer_status f
JOIN dim_geo g ON f.geo_key = g.geo_key
GROUP BY g.country
ORDER BY churn_rate_pct DESC;
        \end{lstlisting}
    \end{block}
    
    \begin{alertblock}{Kết quả}
        \begin{itemize}
            \item \textbf{Germany}: 32\% churn rate (cao nhất!)
            \item \textbf{Spain}: 17\% churn rate
            \item \textbf{France}: 16\% churn rate
        \end{itemize}
    \end{alertblock}
\end{frame}

\begin{frame}{Insights từ OLAP}
    \begin{columns}[t]
        \column{0.5\textwidth}
        \begin{block}{Churn by Geography}
            \begin{figure}
                \centering
                \includegraphics[width=0.95\textwidth]{../reports/figures/churn_by_geography.png}
            \end{figure}
        \end{block}
        
        \column{0.5\textwidth}
        \begin{block}{Churn by Age Group}
            \begin{figure}
                \centering
                \includegraphics[width=0.95\textwidth]{../reports/figures/churn_by_age_group.png}
            \end{figure}
        \end{block}
    \end{columns}
\end{frame}

\begin{frame}{Phân tích sâu hơn}
    \begin{columns}[t]
        \column{0.5\textwidth}
        \begin{block}{Balance by Churn Status}
            \begin{figure}
                \centering
                \includegraphics[width=0.95\textwidth]{../reports/figures/balance_by_churn.png}
            \end{figure}
        \end{block}
        
        \column{0.5\textwidth}
        \begin{block}{Churn by Products}
            \begin{figure}
                \centering
                \includegraphics[width=0.95\textwidth]{../reports/figures/churn_by_products.png}
            \end{figure}
        \end{block}
    \end{columns}
\end{frame}

\begin{frame}{Key Findings & Surprising Insights}
    \begin{alertblock}{Key Findings}
        \begin{itemize}\setlength\itemsep{0em}
            \item Germany có churn rate cao gấp đôi các nước khác
            \item Nhóm Middle-aged (36-55) có churn rate cao nhất
        \end{itemize}
    \end{alertblock}
    
    \vspace{0.3cm}
    
    \begin{alertblock}{Surprising Insights}
        \begin{itemize}\setlength\itemsep{0em}
            \item \textbf{Balance Paradox}: Churned customers có balance cao hơn!
            \item \textbf{Product Overload}: Khách hàng có 3-4 sản phẩm churn nhiều hơn
        \end{itemize}
    \end{alertblock}
\end{frame}

% ============================================================================
% SECTION 5: MACHINE LEARNING
% ============================================================================
\section{Mô hình Machine Learning}

\begin{frame}{ML Pipeline}
    \begin{enumerate}
        \item \textbf{Data Preparation}
        \begin{itemize}
            \item Features: CreditScore, Age, Balance, Geography, Gender, etc.
            \item Target: Exited (0=Retained, 1=Churned)
            \item Train/Test split: 80/20 with stratification
        \end{itemize}
        
        \item \textbf{Preprocessing}
        \begin{itemize}
            \item StandardScaler cho numeric features
            \item OneHotEncoder cho categorical features
            \item Pipeline tự động hóa toàn bộ quy trình
        \end{itemize}
        
        \item \textbf{Models}
        \begin{itemize}
            \item Logistic Regression (baseline)
            \item Random Forest (ensemble)
        \end{itemize}
        
        \item \textbf{Evaluation}
        \begin{itemize}
            \item Accuracy, Precision, Recall, F1-Score
            \item ROC-AUC, Confusion Matrix
        \item Feature Importance
        \end{itemize}
    \end{enumerate}
\end{frame}

\begin{frame}{Kết quả mô hình}
    \begin{columns}[t]
        \column{0.5\textwidth}
        \begin{block}{Logistic Regression}
            \footnotesize
            \vspace{-0.15cm}
            \begin{itemize}\setlength\itemsep{-0.3em}
                \item \textbf{Accuracy}: 81\%
                \item \textbf{ROC-AUC}: 0.85
                \item \textbf{Precision}: 56\%
                \item \textbf{Recall}: 25\%
            \end{itemize}
        \end{block}
        
        \vspace{0.15cm}
        
        \begin{block}{Confusion Matrix}
            \begin{figure}
                \centering
                \includegraphics[width=0.6\textwidth]{../reports/figures/confusion_matrix.png}
            \end{figure}
        \end{block}
        
        \column{0.5\textwidth}
        \begin{block}{Feature Importance}
            \begin{figure}
                \centering
                \includegraphics[width=0.9\textwidth]{../reports/figures/feature_importance.png}
            \end{figure}
        \end{block}
        
        \vspace{0.15cm}
        
        \begin{alertblock}{Top Features}
            \footnotesize
            \vspace{-0.15cm}
            \begin{enumerate}\setlength\itemsep{-0.3em}
                \item Age
                \item NumOfProducts
                \item Balance
                \item Geography
            \end{enumerate}
        \end{alertblock}
    \end{columns}
\end{frame}

% ============================================================================
% SECTION 6: DECISION SUPPORT SYSTEM
% ============================================================================
\section{Hệ hỗ trợ quyết định (DSS)}

\begin{frame}{Kiến trúc DSS}
    \begin{center}
        \begin{tikzpicture}[node distance=1.5cm, auto, scale=0.9, every node/.style={transform shape}]
            % Bottom layer
            \node[rectangle, draw, fill=blue!20, minimum width=3cm, minimum height=1cm] (dwh) {
                \textbf{Data Warehouse}\\
                \tiny Star Schema
            };
            
            % Middle layer
            \node[rectangle, draw, fill=orange!20, minimum width=2cm, minimum height=1cm, above left=of dwh] (olap) {
                \textbf{OLAP Engine}\\
                \tiny SQL Queries
            };
            
            \node[rectangle, draw, fill=orange!20, minimum width=2cm, minimum height=1cm, above right=of dwh] (ml) {
                \textbf{ML Model}\\
                \tiny Predictions
            };
            
            % Top layer
            \node[rectangle, draw, fill=green!20, minimum width=2cm, minimum height=1cm, above=of olap] (viz) {
                \textbf{Visualization}\\
                \tiny Charts
            };
            
            \node[rectangle, draw, fill=green!20, minimum width=2cm, minimum height=1cm, above=of ml] (dashboard) {
                \textbf{Dashboard}\\
                \tiny Interactive UI
            };
            
            % User
            \node[rectangle, draw, fill=red!20, minimum width=5cm, minimum height=1cm, above=2cm of dwh] (user) {
                \textbf{Decision Makers}\\
                \tiny Business Analysts, Managers, Executives
            };
            
            % Arrows
            \draw[->, thick] (dwh) -- (olap);
            \draw[->, thick] (dwh) -- (ml);
            \draw[->, thick] (olap) -- (viz);
            \draw[->, thick] (ml) -- (dashboard);
            \draw[->, thick] (viz) -- (user);
            \draw[->, thick] (dashboard) -- (user);
        \end{tikzpicture}
    \end{center}
\end{frame}

\begin{frame}{Interactive Dashboard}
    \begin{block}{Tính năng Dashboard}
        \begin{itemize}
            \item \textbf{KPI Cards}: Total Customers, Churned, Churn Rate, Avg Balance
            \item \textbf{Interactive Filters}: Country, Age Group, Gender
            \item \textbf{6 Interactive Charts}:
            \begin{multicols}{2}
                \begin{enumerate}
                    \item Churn by Country
                    \item Churn by Age Group
                    \item Balance Distribution
                    \item Churn by Products
                    \item Age Distribution
                    \item Churn by Tenure
                \end{enumerate}
            \end{multicols}
            \item \textbf{Real-time Updates}: Charts cập nhật ngay khi thay đổi filters
        \end{itemize}
    \end{block}
    
    \begin{alertblock}{Công nghệ}
        \textbf{Plotly Dash} - Web framework cho interactive analytics
    \end{alertblock}
\end{frame}

\begin{frame}{Dashboard Screenshots}
    \begin{figure}
        \centering
        \includegraphics[width=0.85\textwidth]{../reports/figures/dashboard_overview.png}
        \caption{\small Dashboard Overview - KPI Cards, Filters, và Charts}
    \end{figure}
\end{frame}

\begin{frame}{Dashboard Screenshots (cont.)}
    \begin{figure}
        \centering
        \includegraphics[width=0.85\textwidth]{../reports/figures/dashboard_bottom.png}
        \caption{\small Dashboard - Phần dưới với các charts tương tác}
    \end{figure}
\end{frame}

\begin{frame}{Ứng dụng DSS trong ra quyết định}
    \begin{block}{Use Case: Phân tích churn ở Germany}
        \begin{enumerate}
            \item \textbf{Bước 1}: Chọn filter Country = "Germany"
            \item \textbf{Bước 2}: KPIs cập nhật → Churn rate = 32\%
            \item \textbf{Bước 3}: Charts hiển thị:
            \begin{itemize}
                \item Age group 46-55 có churn cao nhất
                \item Khách hàng có 3-4 sản phẩm dễ churn
                \item Balance cao nhưng vẫn churn
            \end{itemize}
            \item \textbf{Bước 4}: \textbf{Decision} → Tập trung retention campaign vào:
            \begin{itemize}
                \item Target: Germany, age 46-55, có nhiều sản phẩm
                \item Action: Cải thiện dịch vụ, tăng lãi suất, giảm phí
            \end{itemize}
        \end{enumerate}
    \end{block}
\end{frame}

% ============================================================================
% SECTION 7: RESULTS & CONCLUSION
% ============================================================================
\section{Kết quả \& Kết luận}

\begin{frame}{Kết quả đạt được}
    \begin{columns}[t]
        \column{0.5\textwidth}
        \begin{block}{Về mặt kỹ thuật}
            \small
            \vspace{-0.1cm}
            \begin{itemize}\setlength\itemsep{-0.2em}
                \item $\checkmark$ Star Schema DWH (5 tables)
                \item $\checkmark$ ETL Pipeline tự động
                \item $\checkmark$ 12+ OLAP queries
                \item $\checkmark$ 8+ visualizations
                \item $\checkmark$ ML model (81\% accuracy)
                \item $\checkmark$ Interactive Dashboard
            \end{itemize}
        \end{block}
        
        \column{0.5\textwidth}
        \begin{block}{Về mặt nghiệp vụ}
            \small
            \vspace{-0.1cm}
            \begin{itemize}\setlength\itemsep{-0.2em}
                \item $\checkmark$ Xác định yếu tố chính ảnh hưởng churn
                \item $\checkmark$ Phân khúc khách hàng
                \item $\checkmark$ Dự đoán churn proactive
                \item $\checkmark$ Đề xuất chiến lược retention
                \item $\checkmark$ Hệ thống DSS hoàn chỉnh
            \end{itemize}
        \end{block}
    \end{columns}
    
    \vspace{0.3cm}
    
    \begin{alertblock}{Key Insights}
        \small
        \begin{itemize}\setlength\itemsep{-0.1em}
            \item Germany có churn rate cao nhất (32\%)
            \item Khách hàng 46-55 tuổi dễ churn nhất
            \item Khách hàng có 3-4 sản phẩm có nguy cơ cao
            \item Balance cao không đảm bảo retention
        \end{itemize}
    \end{alertblock}
\end{frame}

\begin{frame}{Hạn chế \& Hướng phát triển}
    \begin{columns}
        \column{0.5\textwidth}
        \begin{block}{Hạn chế}
            \begin{itemize}
                \item Dữ liệu tĩnh (1 snapshot)
                \item Mô hình ML còn đơn giản
                \item Recall cho Churned thấp (25\%)
                \item Chưa deploy production
                \item Thiếu A/B testing
            \end{itemize}
        \end{block}
        
        \column{0.5\textwidth}
        \begin{block}{Hướng phát triển}
            \begin{itemize}
                \item Thu thập time-series data
                \item Thử XGBoost, Neural Networks
                \item Xử lý imbalanced data (SMOTE)
                \item Deploy lên cloud (AWS, GCP)
                \item Tích hợp CRM, email marketing
                \item Automated alerts \& actions
                \item CLV prediction
                \item Next-best-action engine
            \end{itemize}
        \end{block}
    \end{columns}
\end{frame}

\begin{frame}{Tech Stack}
    \begin{block}{Công nghệ sử dụng}
        \small
        \begin{itemize}\setlength\itemsep{-0.1em}
            \item \textbf{Data Warehouse}: Star Schema (CSV/PostgreSQL)
            \item \textbf{ETL}: Python (pandas, numpy)
            \item \textbf{Visualization}: matplotlib, plotly
            \item \textbf{Machine Learning}: scikit-learn
            \item \textbf{Dashboard}: Plotly Dash
            \item \textbf{Database}: PostgreSQL (optional)
            \item \textbf{Notebooks}: Jupyter
        \end{itemize}
    \end{block}
    
    \vspace{0.2cm}
    
    \begin{block}{Project Structure}
        \small
        \begin{itemize}\setlength\itemsep{-0.1em}
            \item \texttt{data/}: raw, interim, processed
            \item \texttt{src/}: data, models, visualization
            \item \texttt{sql/}: DDL, OLAP queries
            \item \texttt{reports/}: LaTeX report + figures
            \item \texttt{notebooks/}: Jupyter notebooks
        \end{itemize}
    \end{block}
\end{frame}

\begin{frame}{Kết luận}
    \begin{block}{Tổng kết}
        \small
        Đề tài đã thành công xây dựng một hệ thống \textbf{Data Warehouse & Decision Support System} hoàn chỉnh cho bài toán Bank Customer Churn, bao gồm:

        \begin{itemize}\setlength\itemsep{-0.1em}
            \item Kho dữ liệu tối ưu với star schema
            \item Quy trình ETL tự động và reproducible
            \item Phân tích OLAP đa chiều với insights giá trị
            \item Mô hình ML dự đoán churn chính xác
            \item Dashboard tương tác cho decision makers
        \end{itemize}
    \end{block}

\end{frame}

\begin{frame}{Kết luận}
    \begin{alertblock}{Giá trị thực tiễn}
        \small
        Hệ thống không chỉ là bài tập học thuật mà còn có thể triển khai thực tế, giúp ngân hàng:
        \begin{itemize}\setlength\itemsep{-0.1em}
            \item Giảm tỷ lệ churn
            \item Tăng customer lifetime value
            \item Ra quyết định dựa trên dữ liệu (data-driven)
        \end{itemize}
    \end{alertblock}
\end{frame}

% ============================================================================
% DEMO SLIDE
% ============================================================================
\begin{frame}{Demo \& Q\&A}
    \begin{center}
        \Huge \textbf{DEMO}
        
        \vspace{0.5cm}
        
        \Large Interactive Dashboard
        
        \texttt{http://localhost:8050}
        
        \vspace{1cm}
        
        \Huge \textbf{Q \& A}
        
        \vspace{0.5cm}
        
        \Large Cảm ơn quý thầy cô đã lắng nghe!
    \end{center}
\end{frame}

% ============================================================================
% REFERENCES
% ============================================================================
\begin{frame}{Tài liệu tham khảo}
    \begin{thebibliography}{9}
        \bibitem{kimball2013}
        Kimball, R., \& Ross, M. (2013). 
        \textit{The Data Warehouse Toolkit: The Definitive Guide to Dimensional Modeling} (3rd ed.). Wiley.
        
        \bibitem{kaggle-churn}
        Kaggle. \textit{Bank Customer Churn Modeling}. 
        \url{https://www.kaggle.com/datasets/shrutimechlearn/churn-modelling}
        
        \bibitem{scikit-learn}
        Pedregosa, F., et al. (2011). 
        \textit{Scikit-learn: Machine learning in Python}. 
        Journal of Machine Learning Research, 12, 2825–2830.
        
        \bibitem{plotly}
        Plotly Technologies Inc. (2015). 
        \textit{Collaborative data science}. 
        \url{https://plot.ly}
    \end{thebibliography}
\end{frame}

% ============================================================================
% BACKUP SLIDES
% ============================================================================
\appendix

\begin{frame}[fragile]{Backup: SQL DDL Example}
    \begin{lstlisting}[language=SQL]
CREATE TABLE dim_customer (
    customer_key SERIAL PRIMARY KEY,
    customer_id INTEGER NOT NULL,
    age INTEGER,
    gender VARCHAR(10),
    tenure INTEGER
);

CREATE TABLE fact_customer_status (
    fact_key SERIAL PRIMARY KEY,
    customer_key INTEGER REFERENCES dim_customer(customer_key),
    time_key INTEGER REFERENCES dim_time(time_key),
    geo_key INTEGER REFERENCES dim_geo(geo_key),
    segment_key INTEGER REFERENCES dim_segment(segment_key),
    balance DECIMAL(15,2),
    estimated_salary DECIMAL(15,2),
    num_of_products INTEGER,
    credit_score INTEGER,
    has_credit_card INTEGER,
    is_active_member INTEGER,
    churn_flag INTEGER
);
    \end{lstlisting}
\end{frame}

\begin{frame}[fragile]{Backup: Python ETL Code}
    \begin{lstlisting}[language=Python]
import pandas as pd

# Extract
df = pd.read_csv('data/raw/Churn_Modelling.csv')

# Transform - Feature Engineering
def categorize_age(age):
    if age < 36: return 'Young'
    elif age < 56: return 'Middle-aged'
    else: return 'Senior'

df['age_group'] = df['Age'].apply(categorize_age)

# Build dimension tables
dim_customer = df[['CustomerId', 'Age', 'Gender', 'Tenure']].copy()
dim_customer = dim_customer.drop_duplicates(subset=['CustomerId'])
dim_customer.reset_index(drop=True, inplace=True)
dim_customer['customer_key'] = dim_customer.index + 1

# Load
dim_customer.to_csv('data/processed/dim_customer.csv', index=False)
    \end{lstlisting}
\end{frame}

\end{document}
