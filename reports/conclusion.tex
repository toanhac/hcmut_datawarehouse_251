\section{Kết luận}

\subsection{Tổng kết}

Đề tài \textbf{"Kho dữ liệu \& Hệ hỗ trợ quyết định cho bài toán Bank Customer Churn"} đã hoàn thành các mục tiêu đề ra, bao gồm:

\begin{enumerate}
    \item \textbf{Thiết kế và triển khai Data Warehouse}:
    \begin{itemize}
        \item Xây dựng thành công star schema với 4 bảng chiều (dim\_customer, dim\_geo, dim\_time, dim\_segment) và 1 bảng sự kiện (fact\_customer\_status).
        \item Schema được tối ưu hóa cho phân tích OLAP, dễ dàng truy vấn và mở rộng.
        \item Cung cấp SQL DDL đầy đủ để triển khai trên PostgreSQL hoặc hệ quản trị cơ sở dữ liệu khác.
    \end{itemize}
    
    \item \textbf{Xây dựng quy trình ETL}:
    \begin{itemize}
        \item Phát triển pipeline ETL tự động bằng Python, xử lý 10,000 bản ghi khách hàng.
        \item Thực hiện feature engineering để tạo các đặc trưng mới (age\_group, income\_group).
        \item Đảm bảo tính toàn vẹn dữ liệu và khả năng tái sử dụng của quy trình.
    \end{itemize}
    
    \item \textbf{Phân tích OLAP và trực quan hóa}:
    \begin{itemize}
        \item Thực hiện 12+ truy vấn OLAP phân tích churn theo nhiều chiều (địa lý, tuổi, thu nhập, sản phẩm).
        \item Tạo 8+ biểu đồ trực quan bằng matplotlib, giúp hiểu rõ đặc điểm của khách hàng churn.
        \item Rút ra được nhiều insights quan trọng (Germany có churn cao, khách hàng có 3-4 sản phẩm dễ churn, v.v.).
    \end{itemize}
    
    \item \textbf{Xây dựng mô hình dự đoán churn}:
    \begin{itemize}
        \item Phát triển mô hình Logistic Regression đạt accuracy ~81\%, ROC-AUC ~0.85.
        \item Thử nghiệm Random Forest để so sánh hiệu suất và phân tích feature importance.
        \item Sử dụng scikit-learn Pipeline để tự động hóa preprocessing và modeling.
    \end{itemize}
    
    \item \textbf{Hệ hỗ trợ quyết định}:
    \begin{itemize}
        \item Tích hợp DWH, OLAP, visualization, và ML thành một hệ thống DSS hoàn chỉnh.
        \item Đề xuất các chiến lược retention cụ thể dựa trên phân tích dữ liệu.
        \item Cung cấp framework để đánh giá hiệu quả của các quyết định.
    \end{itemize}
\end{enumerate}

\subsection{Kết quả đạt được}

\subsubsection{Về mặt kỹ thuật}

\begin{itemize}
    \item \textbf{Data Warehouse}: Star schema với 5 bảng, lưu trữ 10,000+ bản ghi.
    \item \textbf{ETL Pipeline}: Code Python modular, có thể chạy lại và mở rộng dễ dàng.
    \item \textbf{Visualizations}: 8+ biểu đồ PNG chất lượng cao, sẵn sàng cho báo cáo.
    \item \textbf{ML Model}: Accuracy 81\%, ROC-AUC 0.85, đã được lưu và có thể deploy.
    \item \textbf{Documentation}: Code được comment đầy đủ, có README và SQL queries mẫu.
\end{itemize}

\subsubsection{Về mặt nghiệp vụ}

\begin{itemize}
    \item \textbf{Insights}: Xác định được các yếu tố chính ảnh hưởng đến churn (Geography, Age, NumOfProducts).
    \item \textbf{Segmentation}: Phân khúc khách hàng theo nhiều chiều, hỗ trợ targeted marketing.
    \item \textbf{Prediction}: Có thể dự đoán khách hàng nguy cơ cao, can thiệp proactive.
    \item \textbf{Strategy}: Đề xuất 3+ chiến lược retention cụ thể, có thể triển khai ngay.
\end{itemize}

\subsection{Hạn chế của đề tài}

Mặc dù đã đạt được nhiều kết quả tích cực, đề tài vẫn còn một số hạn chế:

\begin{enumerate}
    \item \textbf{Dữ liệu tĩnh}:
    \begin{itemize}
        \item Chỉ có một snapshot dữ liệu tại một thời điểm, không có dữ liệu time-series.
        \item Không thể phân tích trend churn theo thời gian.
        \item Giải pháp: Cần thu thập dữ liệu định kỳ (monthly snapshots) để phân tích xu hướng.
    \end{itemize}
    
    \item \textbf{Mô hình ML còn đơn giản}:
    \begin{itemize}
        \item Chỉ sử dụng Logistic Regression và Random Forest, chưa thử các mô hình advanced (XGBoost, Neural Networks).
        \item Recall cho class Churned còn thấp (~25\%), cần cải thiện.
        \item Giải pháp: Thử nghiệm thêm các mô hình, xử lý imbalanced data bằng SMOTE, điều chỉnh threshold.
    \end{itemize}
    
    \item \textbf{Chưa có UI tương tác}:
    \begin{itemize}
        \item Hiện tại chỉ có Jupyter Notebook và static PNG charts.
        \item Người dùng không thể tương tác (filter, drill-down) trực tiếp.
        \item Giải pháp: Xây dựng web dashboard với Streamlit hoặc Dash.
    \end{itemize}
    
    \item \textbf{Chưa triển khai thực tế}:
    \begin{itemize}
        \item Hệ thống chỉ chạy local, chưa deploy lên server.
        \item Chưa tích hợp với hệ thống CRM hoặc email marketing.
        \item Giải pháp: Deploy model lên cloud (AWS, GCP), tích hợp API.
    \end{itemize}
    
    \item \textbf{Thiếu A/B testing}:
    \begin{itemize}
        \item Chưa thử nghiệm các chiến lược retention trong thực tế.
        \item Không có dữ liệu về hiệu quả thực tế của các đề xuất.
        \item Giải pháp: Cần triển khai pilot program, đo lường ROI.
    \end{itemize}
\end{enumerate}

\subsection{Hướng phát triển}

Để nâng cao chất lượng và tính ứng dụng của hệ thống, một số hướng phát triển trong tương lai:

\begin{enumerate}
    \item \textbf{Mở rộng Data Warehouse}:
    \begin{itemize}
        \item Thêm dimension tables: dim\_product (chi tiết sản phẩm), dim\_channel (kênh giao dịch).
        \item Thu thập dữ liệu định kỳ để có time-series analysis.
        \item Implement slowly changing dimensions (SCD) để track thay đổi của customer attributes.
    \end{itemize}
    
    \item \textbf{Cải thiện mô hình ML}:
    \begin{itemize}
        \item Thử nghiệm XGBoost, LightGBM, CatBoost.
        \item Xử lý imbalanced data bằng SMOTE, ADASYN.
        \item Hyperparameter tuning với Optuna hoặc Bayesian Optimization.
        \item Ensemble methods (Stacking, Voting) để kết hợp nhiều mô hình.
    \end{itemize}
    
    \item \textbf{Xây dựng Real-time Dashboard}:
    \begin{itemize}
        \item Sử dụng Streamlit hoặc Dash để tạo interactive dashboard.
        \item Cho phép user filter theo Geography, Age Group, v.v.
        \item Hiển thị real-time predictions và recommendations.
    \end{itemize}
    
    \item \textbf{Automated Alerts \& Actions}:
    \begin{itemize}
        \item Tự động gửi email/SMS khi phát hiện khách hàng nguy cơ cao.
        \item Tích hợp với CRM để trigger retention campaigns.
        \item Scheduled jobs để chạy ETL và re-train model định kỳ.
    \end{itemize}
    
    \item \textbf{Advanced Analytics}:
    \begin{itemize}
        \item Customer Lifetime Value (CLV) prediction.
        \item RFM (Recency, Frequency, Monetary) segmentation.
        \item Next-best-action recommendation engine.
        \item Causal inference để hiểu tác động thực sự của các yếu tố.
    \end{itemize}
    
    \item \textbf{Deployment \& MLOps}:
    \begin{itemize}
        \item Deploy model lên cloud (AWS SageMaker, GCP Vertex AI).
        \item Implement CI/CD pipeline cho model training và deployment.
        \item Model monitoring để phát hiện model drift.
        \item A/B testing framework để đánh giá hiệu quả của các chiến lược.
    \end{itemize}
\end{enumerate}

\subsection{Kết luận cuối cùng}

Đề tài đã thành công trong việc xây dựng một hệ thống Data Warehouse và Decision Support System hoàn chỉnh cho bài toán Bank Customer Churn. Hệ thống không chỉ giúp phân tích dữ liệu một cách toàn diện mà còn cung cấp khả năng dự đoán và hỗ trợ ra quyết định dựa trên dữ liệu.

Qua quá trình thực hiện, nhóm đã áp dụng được nhiều kiến thức lý thuyết vào thực hành, từ thiết kế dimensional modeling, xây dựng ETL pipeline, phân tích OLAP, trực quan hóa dữ liệu, đến xây dựng mô hình Machine Learning. Đây là một trải nghiệm quý giá, giúp hiểu sâu hơn về cách thức hoạt động của một hệ thống Business Intelligence trong thực tế.

Mặc dù còn một số hạn chế, nhưng với các hướng phát triển đã đề xuất, hệ thống có thể được nâng cấp thành một giải pháp enterprise-grade, sẵn sàng triển khai trong môi trường sản xuất thực tế. Đề tài này không chỉ là một bài tập lớn mà còn là nền tảng để phát triển các dự án Data Warehouse và DSS phức tạp hơn trong tương lai.
