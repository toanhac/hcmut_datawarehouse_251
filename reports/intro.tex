\section{Giới thiệu}

\subsection{Đặt vấn đề}

Trong ngành ngân hàng, việc giữ chân khách hàng hiện tại thường có chi phí thấp hơn nhiều so với việc thu hút khách hàng mới. Theo nghiên cứu, chi phí để có được một khách hàng mới có thể cao gấp 5-25 lần so với việc duy trì một khách hàng hiện tại. Do đó, việc dự đoán và ngăn chặn hiện tượng khách hàng rời bỏ (customer churn) trở thành một ưu tiên hàng đầu đối với các ngân hàng.

Tuy nhiên, để có thể phân tích và dự đoán churn một cách hiệu quả, ngân hàng cần:
\begin{itemize}
    \item Tổ chức dữ liệu khách hàng theo cách tối ưu cho phân tích (Data Warehouse).
    \item Có quy trình xử lý dữ liệu tự động và đáng tin cậy (ETL Pipeline).
    \item Phân tích dữ liệu theo nhiều chiều khác nhau (OLAP).
    \item Xây dựng mô hình dự đoán chính xác (Machine Learning).
    \item Cung cấp thông tin trực quan và dễ hiểu cho ban quản lý (Decision Support System).
\end{itemize}

Đề tài này được thực hiện nhằm xây dựng một hệ thống hoàn chỉnh đáp ứng các yêu cầu trên, áp dụng cho bài toán dự đoán churn của khách hàng ngân hàng.

\subsection{Mục tiêu đề tài}

Mục tiêu chính của đề tài bao gồm:

\begin{enumerate}
    \item \textbf{Thiết kế và triển khai Data Warehouse}: Xây dựng kho dữ liệu theo mô hình ngôi sao (star schema) với các bảng chiều và bảng sự kiện, tối ưu hóa cho phân tích OLAP.
    
    \item \textbf{Xây dựng quy trình ETL}: Phát triển pipeline tự động để trích xuất dữ liệu từ nguồn, làm sạch, biến đổi (bao gồm feature engineering), và nạp vào kho dữ liệu.
    
    \item \textbf{Phân tích OLAP}: Thực hiện các truy vấn phân tích đa chiều để khám phá các mẫu và xu hướng trong dữ liệu khách hàng.
    
    \item \textbf{Trực quan hóa dữ liệu}: Sử dụng Python (matplotlib) để tạo ra các biểu đồ trực quan, giúp hiểu rõ hơn về đặc điểm của khách hàng churn và không churn.
    
    \item \textbf{Xây dựng mô hình dự đoán}: Phát triển mô hình Machine Learning để dự đoán khả năng churn của khách hàng với độ chính xác cao.
    
    \item \textbf{Hệ hỗ trợ quyết định}: Tích hợp các thành phần trên thành một hệ thống DSS hoàn chỉnh, cung cấp thông tin hỗ trợ cho các quyết định kinh doanh.
\end{enumerate}

\subsection{Phạm vi đề tài}

Đề tài tập trung vào:
\begin{itemize}
    \item \textbf{Dữ liệu}: Sử dụng bộ dữ liệu "Bank Customer Churn Modeling" từ Kaggle, bao gồm thông tin về 10,000 khách hàng.
    \item \textbf{Công nghệ}: Python (pandas, numpy, matplotlib, scikit-learn), SQL (PostgreSQL hoặc tương đương).
    \item \textbf{Mô hình DWH}: Star schema với 4 bảng chiều và 1 bảng sự kiện.
    \item \textbf{Mô hình ML}: Logistic Regression và Random Forest cho bài toán phân loại nhị phân.
    \item \textbf{Phân tích}: EDA, OLAP queries, visualization, model evaluation.
\end{itemize}

\subsection{Tổng quan về dữ liệu}

\subsubsection{Nguồn dữ liệu}

Bộ dữ liệu được sử dụng trong đề tài là \textit{Bank Customer Churn Modeling}, có sẵn trên Kaggle. Đây là một bộ dữ liệu mô phỏng thông tin của khách hàng ngân hàng, bao gồm các thuộc tính nhân khẩu học, thông tin tài khoản, và trạng thái churn.

\begin{center}
\begin{tabular}{ |l|l| }
    \hline
    \textbf{Thuộc tính} & \textbf{Thông tin} \\
    \hline
    Nguồn dữ liệu & Kaggle - Bank Customer Churn Modeling \\
    \hline
    Tên file & Churn\_Modelling.csv \\
    \hline
    Số lượng quan sát & 10,000 \\
    \hline
    Số lượng biến & 14 \\
    \hline
    Tỷ lệ churn & ~20\% \\
    \hline
\end{tabular}
\end{center}

\subsubsection{Mô tả các biến}

Bảng dưới đây mô tả chi tiết các biến trong bộ dữ liệu:

\begin{center}
\begin{longtable}{|
>{\arraybackslash}m{.2\textwidth} | >{\centering\arraybackslash}m{.15\textwidth} | >{\arraybackslash}m{.5\textwidth} |}
\hline
\textbf{Tên biến}           & \textbf{Kiểu dữ liệu} & \textbf{Mô tả} \\ \hline
RowNumber                   & Integer               & Số thứ tự của bản ghi (không sử dụng trong phân tích). \\ \hline
CustomerId                  & Integer               & ID duy nhất của khách hàng. \\ \hline
Surname                     & String                & Họ của khách hàng (không sử dụng trong phân tích). \\ \hline
CreditScore                 & Integer               & Điểm tín dụng của khách hàng (300-850). \\ \hline
Geography                   & String                & Quốc gia của khách hàng (France, Germany, Spain). \\ \hline
Gender                      & String                & Giới tính của khách hàng (Male, Female). \\ \hline
Age                         & Integer               & Tuổi của khách hàng. \\ \hline
Tenure                      & Integer               & Số năm khách hàng đã sử dụng dịch vụ ngân hàng (0-10). \\ \hline
Balance                     & Float                 & Số dư tài khoản của khách hàng. \\ \hline
NumOfProducts               & Integer               & Số lượng sản phẩm ngân hàng mà khách hàng đang sử dụng (1-4). \\ \hline
HasCrCard                   & Integer               & Khách hàng có thẻ tín dụng hay không (0=Không, 1=Có). \\ \hline
IsActiveMember              & Integer               & Khách hàng có hoạt động tích cực hay không (0=Không, 1=Có). \\ \hline
EstimatedSalary             & Float                 & Mức lương ước tính của khách hàng. \\ \hline
Exited                      & Integer               & Biến mục tiêu: Khách hàng đã rời bỏ ngân hàng hay chưa (0=Không, 1=Có). \\ \hline
\end{longtable}
\end{center}

\subsubsection{Đặc điểm của dữ liệu}

Một số đặc điểm quan trọng của bộ dữ liệu:

\begin{itemize}
    \item \textbf{Dữ liệu sạch}: Không có giá trị thiếu (missing values), giúp đơn giản hóa quá trình tiền xử lý.
    \item \textbf{Mất cân bằng lớp}: Tỷ lệ khách hàng churn (~20\%) thấp hơn nhiều so với khách hàng không churn (~80\%), cần lưu ý khi đánh giá mô hình.
    \item \textbf{Đa dạng về thuộc tính}: Bao gồm cả biến số (CreditScore, Age, Balance) và biến phân loại (Geography, Gender).
    \item \textbf{Phạm vi giá trị hợp lý}: Các biến đều nằm trong phạm vi thực tế, không có outlier quá cực đoan.
\end{itemize}

\subsection{Cấu trúc báo cáo}

Báo cáo được tổ chức theo các phần sau:

\begin{itemize}
    \item \textbf{Phần 1 - Giới thiệu}: Đặt vấn đề, mục tiêu, phạm vi, và tổng quan về dữ liệu.
    \item \textbf{Phần 2 - Thiết kế kho dữ liệu}: Mô tả chi tiết về star schema, các bảng chiều và bảng sự kiện.
    \item \textbf{Phần 3 - Quy trình ETL}: Trình bày các bước trích xuất, làm sạch, biến đổi, và nạp dữ liệu.
    \item \textbf{Phần 4 - Phân tích OLAP và trực quan hóa}: Các truy vấn phân tích và biểu đồ trực quan.
    \item \textbf{Phần 5 - Mô hình dự đoán churn}: Xây dựng và đánh giá mô hình Machine Learning.
    \item \textbf{Phần 6 - Hệ hỗ trợ quyết định}: Tích hợp các thành phần thành DSS hoàn chỉnh.
    \item \textbf{Phần 7 - Kết luận}: Tổng kết, hạn chế, và hướng phát triển.
\end{itemize}
