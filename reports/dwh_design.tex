\section{Thiết kế kho dữ liệu}

\subsection{Mô hình ngôi sao (Star Schema)}

Kho dữ liệu được thiết kế theo mô hình ngôi sao (star schema), một trong những mô hình phổ biến nhất trong thiết kế Data Warehouse. Mô hình này bao gồm một bảng sự kiện (fact table) ở trung tâm, được kết nối với nhiều bảng chiều (dimension tables) xung quanh.

\textbf{Ưu điểm của Star Schema}:
\begin{itemize}
    \item Đơn giản, dễ hiểu và dễ truy vấn.
    \item Hiệu suất cao cho các truy vấn OLAP.
    \item Tối ưu hóa cho các công cụ Business Intelligence.
    \item Dễ dàng mở rộng khi cần thêm chiều phân tích mới.
\end{itemize}

\subsection{Sơ đồ tổng quan}

Sơ đồ star schema cho bài toán Bank Customer Churn bao gồm:
\begin{itemize}
    \item \textbf{1 bảng sự kiện}: fact\_customer\_status
    \item \textbf{4 bảng chiều}: dim\_customer, dim\_geo, dim\_time, dim\_segment
\end{itemize}

% Có thể thêm hình vẽ sơ đồ star schema ở đây nếu có
% \begin{figure}[h!]
% \centering
% \includegraphics[width=0.8\textwidth]{figures/star_schema.png}
% \caption{Sơ đồ Star Schema cho Bank Customer Churn}
% \label{fig:star_schema}
% \end{figure}

\subsection{Bảng sự kiện: fact\_customer\_status}

Bảng sự kiện lưu trữ các chỉ số (measures) và khóa ngoại tham chiếu đến các bảng chiều.

\subsubsection{Cấu trúc bảng}

\begin{center}
\begin{longtable}{|l|l|p{7cm}|}
\hline
\textbf{Tên cột} & \textbf{Kiểu dữ liệu} & \textbf{Mô tả} \\ \hline
fact\_key & INTEGER (PK) & Khóa chính của bảng sự kiện \\ \hline
customer\_key & INTEGER (FK) & Khóa ngoại tham chiếu đến dim\_customer \\ \hline
time\_key & INTEGER (FK) & Khóa ngoại tham chiếu đến dim\_time \\ \hline
geo\_key & INTEGER (FK) & Khóa ngoại tham chiếu đến dim\_geo \\ \hline
segment\_key & INTEGER (FK) & Khóa ngoại tham chiếu đến dim\_segment \\ \hline
balance & DECIMAL(15,2) & Số dư tài khoản của khách hàng \\ \hline
estimated\_salary & DECIMAL(15,2) & Mức lương ước tính \\ \hline
num\_of\_products & INTEGER & Số lượng sản phẩm ngân hàng \\ \hline
credit\_score & INTEGER & Điểm tín dụng \\ \hline
has\_credit\_card & INTEGER & Có thẻ tín dụng (0/1) \\ \hline
is\_active\_member & INTEGER & Thành viên hoạt động (0/1) \\ \hline
churn\_flag & INTEGER & Trạng thái churn (0=Không, 1=Có) \\ \hline
\end{longtable}
\end{center}

\subsubsection{Grain của bảng sự kiện}

Grain (độ chi tiết) của bảng sự kiện là: \textbf{Một bản ghi cho mỗi khách hàng tại một thời điểm snapshot}.

\subsection{Bảng chiều: dim\_customer}

Bảng chiều khách hàng lưu trữ thông tin nhân khẩu học và đặc điểm của khách hàng.

\begin{center}
\begin{tabular}{|l|l|p{7cm}|}
\hline
\textbf{Tên cột} & \textbf{Kiểu dữ liệu} & \textbf{Mô tả} \\ \hline
customer\_key & INTEGER (PK) & Khóa chính (surrogate key) \\ \hline
customer\_id & INTEGER & ID gốc của khách hàng từ nguồn \\ \hline
age & INTEGER & Tuổi của khách hàng \\ \hline
gender & VARCHAR(10) & Giới tính (Male/Female) \\ \hline
tenure & INTEGER & Số năm sử dụng dịch vụ (0-10) \\ \hline
\end{tabular}
\end{center}

\subsection{Bảng chiều: dim\_geo}

Bảng chiều địa lý lưu trữ thông tin về quốc gia của khách hàng.

\begin{center}
\begin{tabular}{|l|l|p{7cm}|}
\hline
\textbf{Tên cột} & \textbf{Kiểu dữ liệu} & \textbf{Mô tả} \\ \hline
geo\_key & INTEGER (PK) & Khóa chính \\ \hline
country & VARCHAR(50) & Tên quốc gia (France, Germany, Spain) \\ \hline
\end{tabular}
\end{center}

\subsection{Bảng chiều: dim\_time}

Bảng chiều thời gian lưu trữ thông tin về thời điểm snapshot dữ liệu.

\begin{center}
\begin{tabular}{|l|l|p{7cm}|}
\hline
\textbf{Tên cột} & \textbf{Kiểu dữ liệu} & \textbf{Mô tả} \\ \hline
time\_key & INTEGER (PK) & Khóa chính \\ \hline
snapshot\_date & DATE & Ngày snapshot \\ \hline
year & INTEGER & Năm \\ \hline
month & INTEGER & Tháng (1-12) \\ \hline
quarter & INTEGER & Quý (1-4) \\ \hline
\end{tabular}
\end{center}

\subsection{Bảng chiều: dim\_segment}

Bảng chiều phân khúc lưu trữ thông tin về nhóm khách hàng dựa trên tuổi và thu nhập.

\begin{center}
\begin{tabular}{|l|l|p{7cm}|}
\hline
\textbf{Tên cột} & \textbf{Kiểu dữ liệu} & \textbf{Mô tả} \\ \hline
segment\_key & INTEGER (PK) & Khóa chính \\ \hline
age\_group & VARCHAR(20) & Nhóm tuổi (Young, Middle-aged, Senior) \\ \hline
income\_group & VARCHAR(20) & Nhóm thu nhập (Low, Medium, High) \\ \hline
\end{tabular}
\end{center}

\textbf{Quy tắc phân nhóm}:
\begin{itemize}
    \item \textbf{Age Group}:
    \begin{itemize}
        \item Young: 18-35 tuổi
        \item Middle-aged: 36-55 tuổi
        \item Senior: 56+ tuổi
    \end{itemize}
    \item \textbf{Income Group}:
    \begin{itemize}
        \item Low: EstimatedSalary < 50,000
        \item Medium: 50,000 <= EstimatedSalary < 100,000
        \item High: EstimatedSalary >= 100,000
    \end{itemize}
\end{itemize}

\subsection{SQL DDL Schema}

Dưới đây là một phần của SQL DDL để tạo các bảng trong kho dữ liệu:

\begin{lstlisting}[style=sql, caption=Tạo bảng dim\_customer]
CREATE TABLE dim_customer (
    customer_key SERIAL PRIMARY KEY,
    customer_id INTEGER NOT NULL,
    age INTEGER,
    gender VARCHAR(10),
    tenure INTEGER
);
\end{lstlisting}

\begin{lstlisting}[style=sql, caption=Tạo bảng fact\_customer\_status]
CREATE TABLE fact_customer_status (
    fact_key SERIAL PRIMARY KEY,
    customer_key INTEGER REFERENCES dim_customer(customer_key),
    time_key INTEGER REFERENCES dim_time(time_key),
    geo_key INTEGER REFERENCES dim_geo(geo_key),
    segment_key INTEGER REFERENCES dim_segment(segment_key),
    balance DECIMAL(15,2),
    estimated_salary DECIMAL(15,2),
    num_of_products INTEGER,
    credit_score INTEGER,
    has_credit_card INTEGER,
    is_active_member INTEGER,
    churn_flag INTEGER
);
\end{lstlisting}

\textbf{Lưu ý}: File SQL DDL đầy đủ có sẵn tại \texttt{sql/create\_dwh\_schema.sql} trong project.

\subsection{Lợi ích của thiết kế}

Thiết kế star schema này mang lại nhiều lợi ích:

\begin{enumerate}
    \item \textbf{Hiệu suất truy vấn cao}: Các truy vấn OLAP chỉ cần join từ bảng fact đến các bảng dimension, rất nhanh.
    
    \item \textbf{Dễ dàng phân tích đa chiều}: Có thể phân tích churn theo nhiều chiều: thời gian, địa lý, phân khúc khách hàng.
    
    \item \textbf{Tách biệt dữ liệu phân tích và dữ liệu giao dịch}: Không ảnh hưởng đến hệ thống OLTP.
    
    \item \textbf{Hỗ trợ feature engineering}: Bảng dim\_segment chứa các đặc trưng đã được xử lý (age\_group, income\_group), giúp phân tích và modeling dễ dàng hơn.
    
    \item \textbf{Khả năng mở rộng}: Dễ dàng thêm các chiều mới (ví dụ: dim\_product, dim\_channel) khi cần.
\end{enumerate}
